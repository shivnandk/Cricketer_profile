 \documentclass{article}
\usepackage{blindtext}
\usepackage{fancyhdr}
\usepackage{listings}
\usepackage{graphicx}
\usepackage[a4paper, total ={6in ,8in}]{geometry}
%\usepackage[a4paper, left=1in , right = 1in ,bottom=1in,top=1in]{geometry}
\usepackage{graphicx}
\lstdefinestyle{chstyle}
basicstyle=\ttfamily\small
\begin{document}
\pagestyle{fancy}
\fancyhead[L]{{\large\bf{0801CS211080}}}
\fancyhead[R]{{\large\bf{Shivnand Khatri}}}
\section{Cricket Score Board By shivnand}
\section{Aim}
 The aim of this mini Project is to create profile of the cricketers who had played cricket for years and currently playing cricket how much the runs they have scored ,how many wickets they have take theier strike rate economy and there carreer stats.
\section{Functions used in this projects are}
\subsection{ball name}
This function is used to store the name of the batsmen and takes input in string format
\subsection{baller balls}
This function takes the input of the number of balls balled by the baller in his entire carrier and return the number of balls in int type
\subsection{ball over}
This function counts the number of overs balled by the baller in his career and reurn the number of over in double type
\subsection{economy}
This function takes the number of overs as input and calculate the baller economy for his entire carrier
\subsection{ball innings}
This function takes the number of innings the baller balled in his career and return the number of innings in int format
\subsection{ball wicket}
This function takes the number of wickets that a baller takes in his career and return number of wickets in number int format
\subsection{ball wicket average}
This function calculate the the number of runs the baller balled to get a wicket and return the number of ball wicket average in double or decimal format.
\subsection{ball runs}
This function takes the numn=ber of runs the baller did leak in his entire carrier and return the number of runs in int format.
\subsection{ball age}
This functions takes the current age of the baller in int format and return the age of the player .
\subsection{bat name}
This function takes the name of the batsmen and store it and return the name of the batsmen in string format
\subsection{bat strike rate}
This function calculate the career and strike rate of the batsmen return the strike rate of the batsmen in double format.
\subsection{bat average}
This function take the number of innings and runs of the batsmen as the input and calculate the batting average and return the batting average in double format
\subsection{bat runs}
This functions takes the number of runs in int format and store it.
\subsection{bat age}
This function takes the age of the batsmen and store in int format.
\section{bat innings}
This function takes the number of innings the batsmen had played in hi entire career and store it in int format.
\subsection{bar balls faced}
This function take the number of balls the batsmen faced in his entire career and store it in int format and return the balls faced by the batsmen.
\subsection{team name}
This function takes the name of the team in string format and return the name of the team in string format.
\subsection{Profile}
This is one of the most important function of the this function hepls in creating the profile of the players takes the details of the player and store it to its appropriate location.
\subsection{profile disp}
This is another most important function this one displays the information of the player or the profile of the players.
\section{C++ Code}
\begin{lstlisting}[style=chstyle,language=C++]
#include <iostream>
using namespace std;
//Baller class helps in tracking record of the baller
class baller{
    public:
    string ba_name;
    int ba_runs,ba_age,ba_innings,ba_wicket,ba_balls;
    double ba_economy,ba_wicket_average,ba_overs;
    //this function takes the name of the baller
    string ball_name(string name_ip)
    {
        ba_name=name_ip;
        return ba_name;
    }
    //This function takes the number of balls baller balled
    int baller_balls(int ball_ip)
    {
        ba_balls=ball_ip;
        return ba_balls;
    }
    //This function counts th number of the over baller balled
    int ball_over(int ball_ip)
    {
        ba_overs=ball_ip/6.0;
        return ba_overs;
    }
    //This function calculate the economy of the baller
    double economy(int runs_ip){
        ba_economy=(runs_ip/ba_balls)*6;
        return ba_economy;
    }
    //this function track the number of the innings of the baller
    int ball_innings(int innings_ip){
        ba_innings=innings_ip;
        return ba_innings;
    }
    //this function store the age of th e baller
    int ball_age(int age_ip){
        ba_age=age_ip;
        return ba_age;
    }
    //this function store the wickets taken by tyhe baller
    int ball_wicket(int wic_ip){
        ba_wicket=wic_ip;
        return ba_wicket;
    }
    //this function calculate the averge runs per wicket
    double ball_wicket_average(int wic_ip,int ip_runs){
        ba_wicket_average=ip_runs/wic_ip;
        return ba_wicket_average;
    }
    //this function tracks the nuber of runs given by the baller
    int ball_run(int runs_ip){
        ba_runs=runs_ip;
        return ba_runs;
    }
};
//class batsmen tracks the record of the baller
class batsmen{
    public:
    string bt_name;
    double bt_strike_rate,bt_average;
    int bt_runs,bt_age,bt_innings,bt_balls_faced;
    //this function stores the name of the baller
    string bat_name(string name_ip){
        bt_name =name_ip;
        return bt_name;
    }
    //this function calculate the strike rate of the batsmen
    double bat_strike_rate(int ball_faced_ip,int runs_ip){
         bt_strike_rate=((runs_ip/ball_faced_ip)*100.0);
         return bt_strike_rate;
    }
    //this function calculate the average of the batsmen
    double bat_average(int innings_ip,int runs_ip){
        bt_average=runs_ip/innings_ip;
        return bt_average;
    }
    //this function store the runs of the batsmen
    int bat_runs(int runs_ip){
        bt_runs=runs_ip;
        return bt_runs;
    }
    //this function stores the age of the batsmen
    int bat_age(int age_ip){
        bt_age=age_ip;
    }
    //this function stores the number of innings batsmen played
    int bat_innings(int innings_ip){
        bt_innings=innings_ip;
        return bt_innings;
    }
    //this function stores the balls faced by the batsmen
    int bat_balls_faced(int ball_faced_ip){
        bt_balls_faced=ball_faced_ip;
        return bt_balls_faced;
    }
};
//class team keeps the tracks of the team records
class team:public baller,public batsmen{
    public:
    string team_name;
    //this function stores the name of the teamof the player
    string team_names(string team_name_ip){
        team_name=team_name_ip;
        return team_name;
    }
};
//most imp class whole function are here called
class profile{
    public:
    team player[20];
    string name;
    int inputs;
    string type;
    //in this function the profile of the batsmen is made
    void Profile(int Player_no){
    cout<<"Enter the type of the player \n1.batsmen\n2.baller\n3.All-rounder\n";
    cin>>type;
        {
    if(type=="batsmen"){
        cout<<"Enter the name of the batsmen"<<endl;
        std::cin >>name;
        player[Player_no].bat_name(name);
        cout<<"Enter team Name"<<endl;
        std::cin >>name;
        player[Player_no].team_names(name);
        cout<<"Enter batsmen runs scored by the batsmen"<<endl;
        cin>>inputs;
        player[Player_no].bat_runs(inputs);
        cout<<"Enter the age of the batsmen"<<endl;
        cin>>inputs;
        player[Player_no].bat_age(inputs);
        cout<<"Enter the innings played by the batsmen"<<endl;
        cin>>inputs;
        player[Player_no].bat_innings(inputs);
        cout<<"Enter the number of balls faced by the batsmen"<<endl;
        cin>>inputs;
        player[Player_no].bat_balls_faced(inputs);
        }
    if(type=="baller"){
        cout<<"Enter the name of the baller"<<endl;
        std::cin >>name;
        player[Player_no].ball_name(name);
        cout<<"Enter team Name"<<endl;
        std::cin >>name;
        player[Player_no].team_names(name);
        cout<<"Enter the innings played by the baller"<<endl;
        cin>>inputs;
        player[Player_no].ball_innings(inputs);
        cout<<"Enter the age of the baller"<<endl;
        cin>>inputs;
        player[Player_no].ball_age(inputs);
        cout<<"Enter the number of  wickets taken by the baller"<<endl;
        cin>>inputs;
        player[Player_no].ball_wicket(inputs);
        cout<<"Enter number of runs given by the baller"<<endl;
        cin>>inputs;
        player[Player_no].ball_run(inputs);
        cout<<"Enter the number of balls balled by the batsmen"<<endl;
        cin>>inputs;
        player[Player_no].baller_balls(inputs);
    }
    if(type=="All-rounder"){
        cout<<"Enter the name of the All-rounder"<<endl;
        std::cin >>name;
        player[Player_no].bat_name(name);
        cout<<"Enter team Name of the All-rounder"<<endl;
        std::cin >>name;
        player[Player_no].team_names(name);
        cout<<"Enter runs scored by the All-rounder"<<endl;
        cin>>inputs;
        player[Player_no].bat_runs(inputs);
        cout<<"Enter the age of the All-rounder"<<endl;
        cin>>inputs;
        player[Player_no].bat_age(inputs);
        cout<<"Enter the innings played by the All-rounder as batsmen"<<endl;
        cin>>inputs;
        player[Player_no].bat_innings(inputs);
        cout<<"Enter the number of balls faced by the All-rounder"<<endl;
        cin>>inputs;
        player[Player_no].bat_balls_faced(inputs);
        cout<<"Enter the innings played by the All-rounder as baller"<<endl;
        cin>>inputs;
        player[Player_no].ball_innings(inputs);
        cout<<"Enter the number of  wickets taken by the All-rounder"<<endl;
        cin>>inputs;
        player[Player_no].ball_wicket(inputs);
        cout<<"Enter number of runs given by the All-rounder"<<endl;
        cin>>inputs;
        player[Player_no].ball_run(inputs);
        cout<<"Enter the number of balls balled by the All-rounder"<<endl;
        cin>>inputs;
        player[Player_no].baller_balls(inputs);
       }
    }
  }
  //this function display the details of the player
  void profile_disp(int Player_no){
if(type=="batsmen"){
    cout<<"Batsmen Name : "<<player[Player_no].bt_name<<endl;
    cout<<"Team : "<<player[Player_no].team_name<<endl;
    cout<<"Runs : "<<player[Player_no].bt_runs<<endl;
    cout<<"Innings : "<<player[Player_no].bt_innings<<endl;
    cout<<"Age : "<<player[Player_no].bt_age<<endl;
    cout<<"Strike Rate : ";
    cout<<player[Player_no].bat_strike_rate(player[Player_no].bt_balls_faced,player[Player_no].bt_runs)<<endl;
          cout<<"Batting average : ";
          cout<<player[Player_no].bat_average(player[Player_no].bt_innings,player[Player_no].bt_runs)<<endl;
      }
      else if(type=="baller"){
          cout<<"Baller Name : "<<player[Player_no].ba_name<<endl;
          cout<<"Team : "<<player[Player_no].team_name<<endl;
          cout<<"Innings : "<<player[Player_no].ba_innings<<endl;
          cout<<"Age : "<<player[Player_no].ba_age<<endl;
          cout<<"Wickets : "<<player[Player_no].ba_wicket<<endl;
          player[Player_no].ball_wicket_average(player[Player_no].ba_wicket,player[Player_no].ba_runs);
          cout<<"Wicket Average : "<<player[Player_no].ba_wicket_average;
          player[Player_no].economy(player[Player_no].ba_runs);
          cout<<"Economy : "<<player[Player_no].ba_economy;
      }
      else if(type=="All-rounder"){
           cout<<"Batsmen Name : "<<player[Player_no].bt_name<<endl;
          cout<<"Team : "<<player[Player_no].team_name<<endl;
          cout<<"Runs : "<<player[Player_no].bt_runs<<endl;
          cout<<"Innings as Batsmen : "<<player[Player_no].bt_innings<<endl;
          cout<<"Age : "<<player[Player_no].bt_age<<endl;
          cout<<"Strike Rate : ";
          cout<<player[Player_no].bat_strike_rate(player[Player_no].bt_balls_faced,player[Player_no].bt_runs)<<endl;
          cout<<"Batting average : ";
          cout<<player[Player_no].bat_average(player[Player_no].bt_innings,player[Player_no].bt_runs)<<endl;
          cout<<"Innings as Baller : "<<player[Player_no].ba_innings<<endl;
          cout<<"Wickets : "<<player[Player_no].ba_wicket;
           player[Player_no].economy(player[Player_no].ba_runs);
          cout<<"Economy : "<<player[Player_no].ba_economy;
      }
  }
};  
int main()
{
    profile p;
    int total_player;
    cout<< "Enter the number of players of whom you want to create a profile"<<endl;
    cin>>total_player;
    for(int Player=0;Player<total_player;Player++){
    p.Profile(Player);
    p.profile_disp(Player);
    }
    return 0;
}
\end{lstlisting}
\section{C++ output}
\begin{lstlisting}[style=chstyle,language=C++]
Enter the number of players of whom you want to create a profile
1   
Enter the type of the player 
1.batsmen
2.baller
3.All-rounder
batsmen
Enter the name of the batsmen
Virat
Enter team Name
India
Enter batsmen runs scored by the batsmen
20000
Enter the age of the batsmen
35
Enter the innings played by the batsmen
200 
Enter the number of balls faced by the batsmen
20000
Batsmen Name : Virat
Team : India
Runs : 20000
Innings : 200
Age : 35
Strike Rate : 100
Batting average : 100
\end{lstlisting}
\section{Debugging screen shot}
\begin{figure}
    \centering
    \includegraphics[width=1.5\linewidth]{Debugging c++ Output1.png}
    \caption{C++ debug output 1}
    \label{fig:my_label}
\end{figure}
\begin{figure}
    \centering
    \includegraphics[width=1.5\linewidth]{Debugging c++ Output1.png}
    \caption{C++ debug output 1}
    \label{fig:my_label}
\end{figure}

\newpage

\section{Profiling screenshots}\\
\begin{figure}
    \centering
    \includegraphics[width=0.85\linewidth]{p1.jpg}
    \caption{Caption}
    \label{fig:my_label}
\end{figure}

\begin{figure}
    \centering
    \includegraphics{p2.jpg}
    \caption{Caption}
    \label{fig:my_label}
\end{figure}

\begin{figure}
    \centering
    \includegraphics{p3.jpg}
    \caption{Caption}
    \label{fig:my_label}
\end{figure}
\newpage
\section{Java Code}
\begin{lstlisting}[style=chstyle,language=Java]
import java.util.*;
 class Baller{
    public String ba_name;
    public int ba_runs,ba_age,ba_innings,ba_wicket,ba_balls;
    public double ba_economy,ba_wicket_average,ba_overs;
    //this function takes the name of the baller
    public String ball_name(String name_ip)
    {
        ba_name=name_ip;
        return ba_name;
    }
    //This function takes the number of balls baller balled
    public int baller_balls(int ball_ip)
    {
        ba_balls=ball_ip;
        return ba_balls;
    }
    //This function counts th number of the over baller balled
    public double ball_over(int ball_ip)
    {
        ba_overs=ball_ip/6.0;
        return ba_overs;
    }
    //This function calculate the economy of the baller
    public double economy(int runs_ip){
        ba_economy=(runs_ip/ba_balls)*6;
        return ba_economy;
    }
    //this function track the number of the innings of the baller
    public int ball_innings(int innings_ip){
        ba_innings=innings_ip;
        return ba_innings;
    }
    //this function store the age of th e baller
    public int ball_age(int age_ip){
        ba_age=age_ip;
        return ba_age;
    }
    //this function store the wickets taken by tyhe baller
    public int ball_wicket(int wic_ip){
        ba_wicket=wic_ip;
        return ba_wicket;
    }
    //this function calculate the averge runs per wicket
    public double ball_wicket_average(int wic_ip,int ip_runs){
        ba_wicket_average=ip_runs/wic_ip;
        return ba_wicket_average;
    }
    //this function tracks the nuber of runs given by the baller
    public int ball_run(int runs_ip){
        ba_runs=runs_ip;
        return ba_runs;
    }
}
//class batsmen tracks the record of the baller
class Batsmen extends Baller{
    public String bt_name;
    public double bt_strike_rate,bt_average;
    public int bt_runs,bt_age,bt_innings,bt_balls_faced;
    //this function stores the name of the baller
    public String bat_name(String name_ip){
        bt_name =name_ip;
        return bt_name;
    }
    //this function calculate the strike rate of the batsmen
    public double bat_strike_rate(int ball_faced_ip,int runs_ip){
         bt_strike_rate=((runs_ip/ball_faced_ip)*100.0);
         return bt_strike_rate;
    }
    //this function calculate the average of the batsmen
    public double bat_average(int innings_ip,int runs_ip){
        bt_average=runs_ip/innings_ip;
        return bt_average;
    }
    //this function store the runs of the batsmen
    public int bat_runs(int runs_ip){
        bt_runs=runs_ip;
        return bt_runs;
    }
    //this function stores the age of the batsmen
    public int bat_age(int age_ip){
        bt_age=age_ip;
        return bt_age;
    }
    //this function stores the number of innings batsmen played
    public int bat_innings(int innings_ip){
        bt_innings=innings_ip;
        return bt_innings;
    }
    //this function stores the balls faced by the batsmen
    public int bat_balls_faced(int ball_faced_ip){
        bt_balls_faced=ball_faced_ip;
        return bt_balls_faced;
    }
}
//class team keeps the tracks of the team records
class Team extends Batsmen{
    public String team_name;
    //this function stores the name of the teamof the player
    public String team_names(String team_name_ip){
        team_name=team_name_ip;
        return team_name;
    }
}
class Profile{
    Team player[]= new Team[100];
    String name;
    int inputs;
    String type;
    Scanner sc = new Scanner(System.in);
    void profile(int Player_no){
    player[Player_no]=new Team();
    System.out.println
    ("Enter the type of the player\n1.batsmen\n2.baller\n3.All-rounder");
    type=sc.nextLine();
    System.out.println(type);
    {
        switch(type){
    case "batsmen":
        {
        System.out.println("Enter the name of the batsmen");
        name=sc.nextLine();
        player[Player_no].bat_name(name);
        System.out.println("Enter team Name");
        name=sc.nextLine();
        player[Player_no].team_names(name);
        System.out.println("Enter runs scored by the batsmen");
        inputs=sc.nextInt();
        player[Player_no].bat_runs(inputs);
        System.out.println("Enter the age of the batsmen");
        inputs=sc.nextInt();
        player[Player_no].bat_age(inputs);
        System.out.println("Enter the innings played by the batsmen");
        inputs=sc.nextInt();
        player[Player_no].bat_innings(inputs); 
        System.out.println("Enter the number of balls faced by the batsmen");
        inputs=sc.nextInt();
        player[Player_no].bat_balls_faced(inputs);
        break;
        }
    case "baller":
        {
        System.out.println("Enter the name of the baller");
        name=sc.nextLine();
        player[Player_no].ball_name(name); 
        System.out.println("Enter team Name");
        name=sc.nextLine();
        player[Player_no].team_names(name);
        System.out.println("Enter the innings played by the baller"); 
        inputs=sc.nextInt();
        player[Player_no].ball_innings(inputs);
        System.out.println("Enter the age of the baller");
        inputs=sc.nextInt();
        player[Player_no].ball_age(inputs);
        System.out.println("Enter the number of  wickets taken by the baller");
        inputs=sc.nextInt();
        player[Player_no].ball_wicket(inputs);
        System.out.println("Enter number of runs given by the baller" );
        inputs=sc.nextInt();
        player[Player_no].ball_run(inputs);
        System.out.println("Enter the number of balls balled by the baller");
        inputs=sc.nextInt();
        player[Player_no].baller_balls(inputs);
        break;
        }
    case "All-rounder":
    {
    System.out.println("Enter the name of the All-rounder");
    name=sc.nextLine();
    player[Player_no].bat_name(name); 
    System.out.println("Enter team Name of the All-rounder");
    name=sc.nextLine();
    player[Player_no].team_names(name);
    System.out.println("Enter runs scored by the All-rounder");
    inputs=sc.nextInt();
    player[Player_no].bat_runs(inputs);
    System.out.println("Enter the age of the batsmen");
    inputs=sc.nextInt();
        player[Player_no].bat_age(inputs);
    System.out.println("Enter the innings played by the All-rounder as batsmen");
    inputs=sc.nextInt();
    player[Player_no].bat_innings(inputs);
     System.out.println("Enter the number of balls faced by the All-rounder");
    inputs=sc.nextInt();
    player[Player_no].bat_balls_faced(inputs);
    System.out.println("Enter the innings played by the All-rounder as baller");
    inputs=sc.nextInt();
    player[Player_no].ball_innings(inputs);
    System.out.println("Enter the number of  wickets taken by the All-rounder");
    inputs=sc.nextInt();
    player[Player_no].ball_wicket(inputs);
    System.out.println("Enter number of runs given by the All-rounder");
    inputs=sc.nextInt();
    player[Player_no].ball_run(inputs);
    System.out.println("Enter the number of balls balled by the All-rounder");
    inputs=sc.nextInt();
    player[Player_no].baller_balls(inputs);
    }
    }
    }
        
    }
    void profile_disp(int Player_no){
        switch(type)
        {
            case "baller":
        {
        System.out.println("Baller Name : "+player[Player_no].ba_name);
        System.out.println("Team : "+player[Player_no].team_name);
        System.out.println("Innings : "+player[Player_no].ba_innings);
        System.out.println("Age : "+player[Player_no].ba_age);
        System.out.println("Wickets : "+player[Player_no].ba_wicket);
        player[Player_no].ball_wicket_average
        (player[Player_no].ba_wicket,player[Player_no].ba_runs);
        System.out.println("Wicket Average : "+player[Player_no].ba_wicket_average);
        player[Player_no].economy(player[Player_no].ba_runs);
        System.out.println("Economy : "+player[Player_no].ba_economy);
        }
        case "All-rounder":
        {
        System.out.println("All-rounder Name : "+player[Player_no].bt_name);
        System.out.println("Team : "+player[Player_no].team_name);
        System.out.println("Runs : "+player[Player_no].bt_runs);
        System.out.println("Innings as Batsmen : "+player[Player_no].bt_innings);
        System.out.println("Age : "+player[Player_no].bt_age);
        System.out.print("Strike Rate : ");
        System.out.println(player[Player_no].
        bat_strike_rate(player[Player_no].bt_balls_faced,
        player[Player_no].bt_runs));
        System.out.print("Batting average : ");
        System.out.println(player[Player_no].
        bat_average(player[Player_no].bt_innings,player[Player_no].bt_runs));
        System.out.println("Innings as Baller : "+player[Player_no].ba_innings);
        player[Player_no].
        ball_wicket_average(player[Player_no].
        ba_wicket,player[Player_no].ba_runs);
        System.out.println("Wickets : "+player[Player_no].ba_wicket);
         player[Player_no].economy(player[Player_no]
         .ba_runs);
        System.out.println("Economy : "+player[Player_no].ba_economy);
        }
        case "batsmen":
        {
        System.out.println("Batsmen Name : "+player[Player_no].bt_name);
        System.out.println("Team : "+player[Player_no].team_name);
        System.out.println("Runs : "+player[Player_no].bt_runs);
        System.out.println("Innings : "+player[Player_no].bt_innings);
        System.out.println("Age : "+player[Player_no].bt_age);
        System.out.print("Strike Rate : ");
        System.out.println(player[Player_no].
        bat_strike_rate(player[Player_no].bt_balls_faced,player
        [Player_no].bt_runs));
        System.out.print("Batting average : ");
        System.out.println(player[Player_no].bat_average(player[Player_no].
        bt_innings,player[Player_no].
        bt_runs));
        }
    }
    }
}
class Main{
    public static void main(String args[]){
        Scanner sc = new Scanner(System.in);
        Profile p = new Profile();
        int total_player;
        System.out.println
        ("Enter the number of players of whom you want to create a profile");
        total_player=sc.nextInt();
        for(int Player=0;Player<total_player;Player++){
         p.profile(Player);
         p.profile_disp(Player);
        }
    }
}

\begin{lstlisting}

Enter the number of players of whom you want to create a profile
1
Enter the type of the player
1.batsmen
2.baller
3.All-rounder
batsmen
batsmen
Enter the name of the batsmen
shashank
Enter team Name
India
Enter runs scored by the batsmen
3000
Enter the age of the batsmen
32
Enter the innings played by the batsmen
100 
Enter the number of balls faced by the batsmen
3000
Batsmen Name : shashank
Team : India
Runs : 3000
Innings : 100
Age : 32
Strike Rate : 100.0
Batting average : 30.0
\end{lstlisting}
\section{Java Profile output}
\newpage

\begin{figure}
    \centering
    \includegraphics[width=0.5\linewidth]{java_profile1.jpg}
    \caption{Caption}
    \label{fig:my_label}
\end{figure}
\begin{figure}
    \centering
    \includegraphics[width=0.5\linewidth]{java_profile2.jpg}
    \caption{Caption}
    \label{fig:my_label}
\end{figure}
\begin{figure}
    \centering
    \includegraphics[width=0.5\linewidth]{java_profile3.jpg}
    \caption{Caption}
    \label{fig:my_label}
\end{figure}
\begin{figure}
    \centering
    \includegraphics[width=0.5\linewidth]{java_profile4.jpg}
    \caption{Caption}
    \label{fig:my_label}
\end{figure}
\begin{figure}
    \centering
    \includegraphics[width=0.5\linewidth]{java_profile5.jpg}
    \caption{Caption}
    \label{fig:my_label}
\end{figure}
\begin{figure}
    \centering
    \includegraphics[width=0.5\linewidth]{java_profile6.jpg}
    \caption{Caption}
    \label{fig:my_label}
\end{figure}

\newpage


\section{Debugging Screenshot Java}
\begin{figure}
    \centering
    \includegraphics{debug_java_output.2jpg.jpg}
    \caption{Caption}
    \label{fig:my_label}
\end{figure}
\begin{figure}
    \centering
    \includegraphics{debug_java_output.jpg}
    \caption{Caption}
    \label{fig:my_label}
\end{figure}
\begin{figure}
    \centering
    \includegraphics{debug_java_output4.jpg}
    \caption{Caption}
    \label{fig:my_label}
\end{figure}
\end{document}